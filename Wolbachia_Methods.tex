\documentclass[11pt]{article}
\linespread{1.5}
%\usepackage{natbib}
\usepackage[left=1in,right=1in,top=1in,bottom=1in]{geometry}
\usepackage{gensymb}
\usepackage{fancyhdr}
\usepackage{pifont}
\usepackage{graphicx}
\usepackage{mdwlist}
\pagestyle{fancy}
\usepackage{amsmath}



\begin{document}
\raggedright
\setlength{\parindent}{0.3in}

\section{Methods}
\subsection{Motivating data and previous analysis}
Blah blah blah Ahmed et al. 2015 data from 37 studies that screened \emph{Wolbachia} in Lepidopetera comprising 300 species. 

Previously Ahmed et al. 2015 used a likelihood-based approach to describe the distribution of \emph{Wolbachia} infection in Lepidoptera. Specifically, they used a beta-binomial model to estimate the mean proportion of individuals infected within a given species. They used the same distribution to calculate the incidence of infection as well, where incidence was the proportion of species infected above a threshold frequency $c$ (e.g., one infection in 1000 individuals, or 0.001). 

\subsection{Bayesian hierarchical models}
In contrast to Ahmed et al, we adopted a hierachical Bayesian approach to estimate the probability of infection. 

\bibliography{Zach}
%\renewcommand*{\bibname}{Literature Cited}
 \bibliographystyle{amnat}

\end{document}

