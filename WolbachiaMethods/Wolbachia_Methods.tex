\documentclass[12pt]{article}
\linespread{1.5}
\usepackage{natbib}
\usepackage[left=1in,right=1in,top=1in,bottom=1in]{geometry}
\usepackage{gensymb}
\usepackage{fancyhdr}
\usepackage{pifont}
\usepackage{graphicx}
\usepackage{mdwlist}
\pagestyle{fancy}
\usepackage{amsmath}
\usepackage{lineno}
  \pagewiselinenumbers
  \modulolinenumbers[2]
\usepackage{setspace}
  \doublespacing



\begin{document}
\raggedright
\setlength{\parindent}{0.3in}

\section{Methods}
\subsection{Motivating data and previous analysis}

\cite{Ahmed:2015aa} 2015 used a likelihood-based approach to describe the distribution of \emph{Wolbachia} infection in Lepidoptera. Using the data from 37 previous studies, the dataset included over 300 species from 17 families across the order. \cite{Ahmed:2015aa} used a beta-binomial model to estimate the mean proportion of individuals infected within a given species \citep{Hilgenboecker:2008aa}. They used the same distribution to calculate the incidence of infection as well, where incidence was the proportion of species infected above a threshold frequency $c$ (i.e.,, one infection in 1000 individuals, or 0.001) \citep{Weinert:2015aa}. 

\subsection{Bayesian hierarchical models}
	In contrast to \cite{Ahmed:2015aa}, we adopted a hierachical Bayesian approach to estimate the probability of infection prevalence within and among species of Lepidoptera. Each observation ($N=661$)---the number of \emph{Wolbachia-infected} individuals---was nested within species ($S=301$) and  modeled as:

\begin{equation}
	infected_{ij} \sim \mathrm{Binomial}(\theta_{j}, n_{i}).
\end{equation}

where $i = 1, 2, \ldots, 661$ and $J = 1, 2, \ldots, 301$. Here $infected_{i,j}$ indicates the number of infected individuals from the $i$th observation of the $j$th species,  $n_{i}$ is the total number of screened insects in observation $i$, and $\theta_{j}$ is the probability of infection for species $j$. 

	We then assumed that the species-level probabilities of infection were drawn from a common normal distribution with a mean ($\pi$) and standard deviation ($\sigma$). The normal distribution is unconstrained, but $\theta$ is bounded between zero and one. Therefore we used the inverse-logit function to transform each $\theta_{j}$ to the [0,1] interval: 
    
\begin{align}    
	\theta_{j} &= \frac{1}{1 + \exp(-\alpha_{j})} \\
    \alpha_{j} &\sim \mathrm{Normal}((logit(\pi), \sigma).
\end{align}

The mean ($\pi$) describes the overall probability of infection across the lepidopteran species studied, and we gave it an uninformative beta hyperprior. To convert $\pi$ to an unconstrained scale for use with the normal prior, we transformed it using the logit function. Thus:

\begin{align}
  \pi &\sim \mathrm{Beta}(1,1)\\
  logit(\pi) &= \log \frac{\pi}{1-\pi}.
\end{align}

Likewise, the standard deviation ($\sigma$) measures how much variation in the probability of infection there is across species. If $\sigma$ is small, then infection probabilities will be similar among species. Conversely, if $\sigma$ is large, species-specific probabilities of infection will be more idiosyncratic. We chose a weakly informative prior for $\sigma$ as recommended by ***Gelman as follows:

\begin{equation}
	\sigma \sim \mathrm{half-Cauchy}(0,2).
\end{equation}





\bibliography{Wolbachia}
%\renewcommand*{\bibname}{Literature Cited}
 \bibliographystyle{amnat}

\end{document}

